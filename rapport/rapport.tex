\documentclass[french,a4paper]{article}
\usepackage[utf8]{inputenc}
\usepackage[T1]{fontenc}
\usepackage[frenchb]{babel}

\title{FMIN327 Cognition individuelle et collective\\ Protocoles artificiels entre agents naturels}
\author{BONAVERO Yoann \and DUPÉRON Georges}

\begin{document}

\maketitle
\tableofcontents
\newpage

\section{Introduction}
\subsection{approche générale}
Tout individu quel qu'il soit, privé de toutes formes de communication, 
d'émotions et de sensations, ne peuvent en aucune manière évoluer et 
former de groupes cohérents. L'intégrité et la cohérence d'un groupe 
passe majoritairement par un échange d'informations entre les individus.
Celles-ci ne peuvent pas être transmises n'importe comment, les individus 
constituant le groupe doivent être en mesure de les comprendre. 

Le formatage de l'information devient essentiel tout comme le 
support qui va être utilisé pour la transmettre.
Au fil du temps les individus ont apris à échanger des idées et des 
concepts de diverses manières. Que ce soit par le biais de gestes, de 
dessin, de rictus ou bien d'autre, les hommes ont petit à petit mis en 
place un moyen de communication efficace. Toutes ces façons de 
transmettre l'information ont sans cesse évolué pour répondre en 
permanance aux besoins.

Dans la communication il est possible de regrouper en deux grandes 
catégorie les protocoles de communication. Il y a ceux qui sont 
"naturels" et ceux qui sont engendré par un "individu" dit artificiels.

\subsection{Les limites}
Dans cette présentation nous nous limiterons aux seul protocoles 
artificiel et plus précisément aux protocoles artificiels formels. En 
d'autre termes ceux qui ont normes, règles bien définis, qui permettent 
de définir ce protocole de manière unique et sans ambiguïtées.

Les autres limites sont celles définies par le sujet lui même 
"protocoles artificiel" et "agent naturels". Nous de traiterons pas de 
la communication entre agent artificiels comme les robots.

\newpage
% L'espéranto.
\section{Une langue artificielle : L'Espéranto}
\subsection{Présentation}
\subsection{Origine et objectif}
\subsection{Principe de fonctionnement}
\subsection{...}

% Alphabet et supports
\section{Alphabets et supports}
\subsection{L'alphabet}
\subsubsection{Ses origines}
\subsubsection{Dans quel but}
\subsubsection{Des exemples}

\subsection{Les supports}
\subsubsection{Le morse}
Le code Morse est généralement attribué à Samuel Morse. Ce code à été inventé pour la télégraphie en 1835.
Il consiste en une série d'impulsions. Les lettres, chiffres, signes de ponctuation sont représenté par des séries d'impulsions.
Seulement deux types d'impulsions son nécessaires pour tout coder, une impulsion courrte que l'on appelle généralement
"Point" et une impulsion longue appellée "Trait".

Co code possède un très faible expréssivité du fait d'un nombre important d'impulsions utilisées pour un seul caractère.
Ce code est considéré comme le précurseur des communications numériques que l'on connait.

Les militaire ont utilisé ce code pour effectuer des transmission codées, et même si un spectre de fréquence radio et toujours 
réservé pour les sueles émission en morse, ce code n'apporte pas de grand intérêt en terme de communication homme-machine.

\subsubsection{Le braille}
Le braille est une manière de représenter l'alphabet. il consiste en une représentation en relief de l'ensemble des lettres, chiffres, ponctuation, sumboles etc
en relief. Il a été étudie pour permettre un lecture simplement avec les doigts. Le braille a été mis au point par Louis Braille en 1824 et 
reste aujourd'hui après une série de réformes et normalisation toujourstrès utilisé.

Ce code est un peu plus expressif que le code Morse vu précédement puisqu'il permet de représenter la majorité est symboles par
une seule cellule. Cependant l'utilité dans l'échange homme-machine reste, comme pour le Morse, très peu utile.


\subsubsection{La langue des signes}

\subsubsection{Les vues}
Les vues regroupes un grand nombre de représentation très formelle d'objets ou éléments sur un support. Qu'elle soit
en perspective, d'ensemble, de coupe, éclatée etc, les vues sont normalisées et laissent place à très peu, voire aucune ambigüité dans les
représentations.

De ce fait il est assez simple de stocker sur un machine ce genre d'informations et même de les restituer, par exemple sous
forme visuelle (sur un écran).

\subsubsection{Les formules mathématiques}
Les formules mathématiques représente une des représentation les plus normalisées, même si les personnes ont tendance à 
adapter les représentation, symboles mathématiques à leurs besoins.
Les formules mathématiques ont une "expressivité" relativement importante. Leur inconvénient majoritaire se trouve au niveau 
du lien entre une formule et son contexte.
En effet il est très difficile de retrouver le contexte d'une formule lorsque celle-ci est déjà écrite et isolée.


\section{conclusion}
Graphique de comparaison des différents supports en terme d'expressivité et de formalismem.

\end{document}
