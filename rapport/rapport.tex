\documentclass[french,a4paper]{article}
\usepackage[utf8]{inputenc}
\usepackage[T1]{fontenc}
\usepackage[frenchb]{babel}

\title{FMIN327 Cognition individuelle et collective\\ Protocoles artificiels entre agents naturels}
\author{BONAVERO Yoann \and DUPÉRON Georges}

\begin{document}

\maketitle
\tableofcontents
\newpage

\section{Introduction}
\subsection{approche générale}
Tout individu quel qu'il soit, privé de toutes formes de communication, 
d'émotions et de sensations, ne peuvent en aucune manière évoluer et 
former de groupes cohérents. L'intégrité et la cohérence d'un groupe 
passe majoritairement par un échange d'informations entre les individus.
Celles-ci ne peuvent pas être transmises n'importe comment, les individus 
constituant le groupe doivent être en mesure de les comprendre. 

Le formatage de l'information devient essentiel tout comme le 
support qui va être utilisé pour la transmettre.
Au fil du temps les individus ont apris à échanger des idées et des 
concepts de diverses manières. Que ce soit par le biais de gestes, de 
dessin, de rictus ou bien d'autre, les hommes ont petit à petit mis en 
place un moyen de communication efficace. Toutes ces façons de 
transmettre l'information ont sans cesse évolué pour répondre en 
permanance aux besoins.

Dans la communication il est possible de regrouper en deux grandes 
catégorie les protocoles de communication. Il y a ceux qui sont 
"naturels" et ceux qui sont engendré par un "individu" dit artificiels.

\subsection{Les limites}
Dans cette présentation nous nous limiterons aux seul protocoles 
artificiel et plus précisément aux protocoles artificiels formels. En 
d'autre termes ceux qui ont normes, règles bien définis, qui permettent 
de définir ce protocole de manière unique et sans ambiguïtées.

Les autres limites sont celles définies par le sujet lui même 
"protocoles artificiel" et "agent naturels". Nous de traiterons pas de 
la communication entre agent artificiels comme les robots.

\newpage
% L'espéranto.
\section{Une langue artificielle : L'Espéranto}
\subsection{Présentation}
\subsection{Origine et objectif}
\subsection{Principe de fonctionnement}
\subsection{...}

% Alphabet et supports
\section{Alphabets et supports}
\subsection{L'alphabet}
\subsubsection{Ses origines}
\subsubsection{Dans quel but}
\subsubsection{Des exemples}

\subsection{Les supports}
\subsubsection{Le morse}
\subsubsection{Le braille}
\subsubsection{La langue des signes}


\section{conclusion}

\end{document}
