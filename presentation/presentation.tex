\documentclass{beamer}
\usepackage[utf8]{inputenc}
\usepackage[frenchb]{babel}
\usepackage{tikz}
\makeatletter\def\@makecaption{}\makeatother
\usepackage[scriptsize]{caption}
\renewcommand*{\figurename}{}
\usetikzlibrary{shapes,positioning,snakes,calc,chains}
\usetheme{Frankfurt}
\usepackage{graphicx}

\title{FMIN327 Cognition individuelle et collective\\ Protocoles artificiels entre agents naturels}
\author{DUPÉRON Georges \and\\ BONAVERO Yoann}
\institute{Université Montpellier II,\\Département informatique  \\ Master 2 IFPRU \\ Sous la direction de Monsieur Jacques Ferber}
\date{Jeudi, 3 novembre 2011}

\defbeamertemplate*{footline}{shadow theme}
{%
  \leavevmode%
  \hbox{\begin{beamercolorbox}[wd=.5\paperwidth,ht=2.5ex,dp=1.125ex,leftskip=.3cm plus1fil,rightskip=.3cm]{author in head/foot}%
    \usebeamerfont{author in head/foot}\insertframenumber\,/\,\inserttotalframenumber%\hfill\url{http://www.pticlic.fr/}
  \end{beamercolorbox}%
  \begin{beamercolorbox}[wd=.5\paperwidth,ht=2.5ex,dp=1.125ex,leftskip=.3cm,rightskip=.3cm plus1fil]{title in head/foot}%
    \usebeamerfont{title in head/foot}\insertshorttitle%
  \end{beamercolorbox}}%
  \vskip0pt%
}

\AtBeginSection[] { 
  \begin{frame}
    \frametitle{Plan} 
    \tableofcontents[currentsection] 
  \end{frame} 
  \addtocounter{framenumber}{-1} 
}

\begin{document}
\renewcommand*{\figurename}{}

\begin{frame}
  \titlepage
\end{frame}

\section{Introduction}

\begin{frame}
  \frametitle{Introduction}
  \begin{block}{}
    Pour se comprendre, il faut des concepts et un mécanisme d'échange communs.
  \end{block}
  \begin{figure}
    \centering
    \begin{tikzpicture}[scale=0.8,every node/.style={transform shape}]
      \begin{scope}
        \node[draw,rectangle,minimum width=3cm,minimum height=2cm,anchor=south west] (ra) at (0,0) {};
        \path[fill=black!50!white](0.7cm,0.7cm) circle (0.3cm);
        \path[fill=green!80!black]  (1.5cm,1.3cm) circle (0.3cm);
        \path[fill=orange] (2.3cm,0.7cm) circle (0.3cm);
        \node[draw,fill=green!30!red,rectangle,minimum width=0.3cm,minimum height=0.3cm] (la1) at (ra.15) {};
        \node[draw,fill=green!50!blue,rectangle,minimum width=0.3cm,minimum height=0.3cm] (la2) at (ra.-15) {};
      \end{scope}
      \begin{scope}[xshift=6cm]
        \node[draw,rectangle,minimum width=3cm,minimum height=2cm,anchor=south west] (rb) at (0,0) {};
        \path[fill=red]   (1cm,1cm) circle (0.3cm);
        \path[fill=blue]  (2cm,1cm) circle (0.3cm);
        \node[draw,fill=white!20!blue!70!red,rectangle,minimum width=0.3cm,minimum height=0.3cm] (lb1) at (rb.165) {};
        \node[draw,fill=green!50!blue,rectangle,minimum width=0.3cm,minimum height=0.3cm] (lb2) at (rb.195) {};
      \end{scope}
      \begin{scope}[xshift=3cm,yshift=3cm]
        \node[draw,rectangle,minimum width=3cm,minimum height=2cm,anchor=south west] (rc) at (0,0) {};
        \path[fill=red]   (0.7cm,0.7cm) circle (0.3cm);
        \path[fill=green!80!black] (1.5cm,1.3cm) circle (0.3cm);
        \path[fill=blue]  (2.3cm,0.7cm) circle (0.3cm);
        \node[draw,fill=white!20!blue!70!red,rectangle,minimum width=0.3cm,minimum height=0.3cm] (lc1) at (rc.270) {};
      \end{scope}
      
      \only<2->{\draw[<->, thick] (lb1)--(lc1);}
      \only<3->{\draw[<->, thick, dashed, red] (la2)--(lb2);}
      \only<4->{\draw[<->, thick, dashed, red] (la1)--(lc1);}
    \end{tikzpicture}
    \caption{Communication entre trois agents}
  \end{figure}
\end{frame}

\begin{frame}
\frametitle{Introduction}
\begin{block}{Définitions}
\begin{itemize}
	\item Protocole artificiel\\
	Règles arbitraires qui codifient l'échange entre des agents.\\
	Nous étudierons seulement les protocoles artificiels formels.
	\item Agent naturel\\
	Entité capable d'interagir non crée par l'homme.
\end{itemize}
\end{block}
\end{frame}

\begin{frame}
  \frametitle{Plan} 
  \tableofcontents
\end{frame} 

\section{Espéranto}

\begin{frame}
\frametitle{Espéranto}
\begin{block}{L'espéranto, qu'est-ce que c'est ?}
\begin{itemize}
\item Langue universelle.
\item Une langue dite \emph{construite}.
\item Une langue vivante.
\end{itemize}
\end{block}
\end{frame}

\begin{frame}
\frametitle{Espéranto}
\begin{itemize}
\item Une grammaire régulière.
\begin{itemize}
\item Fari : mi faras, vi faras, \dots
\item Esti : mi estas, vi estas, \dots
\end{itemize}
\item Une langue agglutinante.
\begin{itemize}
\item {\color{brown}\emph{vid}}i = voir
\item {\color{brown}\emph{vid}}o = vue
\item {\color{brown}\emph{vid}}ebla = visible
\item ne{\color{brown}\emph{vid}}ebla = invisible
\end{itemize}
\end{itemize}
\end{frame}

\section[Contrats]{Législation et contrats formels}

\begin{frame}
  \frametitle{Législation et contrats formels}% GovMil.pdf
  \begin{block}{Business Contract Language (BCL)}
    \begin{itemize}
    \item Formalisation des obligations, permissions, pénalités.% en cas de violation des obligations
    \item Vérification de la consistance.
    \item Plusieurs tentatives ont été faites.
    \end{itemize}
  \end{block}
  \vskip 1em
  \footnotesize
\texttt{Policy: MakeGoodsAvailable\\
\quad Role: Supplier\\
\quad Modality: Obligation\\
\quad Trigger: PurchaseOrder\\
\quad Behaviour: GoodsAvailable.date before (PurchaseOrder.date + 1)}
\end{frame}

\section[Alphabets]{Alphabets et notations}

\begin{frame}
  \frametitle{Alphabets et Logogrammes}
  \begin{block}{Alphabets, phonogrammes et syllabaires}
  \begin{itemize}
  \item Alphabet roman
  \item Alphabet grec
  \item \dots
  \end{itemize}
  \end{block}
  \begin{block}{Idéogrammes et pictogrammes}
  % TODO : images
  \begin{itemize}
  \item Idéogrammes chinois
  \item Hiéroglyphes
  \item \dots
  \end{itemize}
  \end{block}
\end{frame}

\begin{frame}
  \frametitle{Le code morse}
  \begin{itemize}
  \item Invention attribué à Samuel Morse (1791 - 1872)
  \item Créé en 1835
  \item Seulement deux types d'impulsion sont nécessaires.
  \end{itemize}
  % Ajouter la photo de Samuel Morse.
  
  \begin{figure}
    \centering
    \begin{tikzpicture}
      \begin{scope}[
        start chain,
        node distance=1mm,
        every node/.style={on chain},
        dot/.style={fill=black,circle,minimum height=1mm,minimum width=1mm,inner sep=0pt},
        dash/.style={fill=black,rounded rectangle,minimum height=1mm,minimum width=4mm,inner sep=0pt}]
        \node[dot] {}; \node[dash] {}; \node[dot] {}; \node[dash] {}; \node[dot] {}; \node[dot] {}; \node[dot] {}; \node[dash] {};
        \node[dash] {}; \node[dot] {}; \node[dot] {}; \node[dot] {}; \node[dot] {}; \node[dot] {}; \node[dot] {}; \node[dash] {};
        \node[dash] {}; \node[dot] {}; \node[dot] {}; \node[dot] {}; \node[dot] {}; \node[dash] {};
      \end{scope}
    \end{tikzpicture}
    \caption{Mot «Alphabet» écrit en morse.}
  \end{figure}
\end{frame}


\begin{frame}  
  \frametitle{Le braille}
  \begin{center}
  % TODO
  \includegraphics[width=4cm]{./include/cellule_braille.jpg}
  \end{center}
  \normalsize \begin{itemize}
  \item Louis Braille (1809 - 1852)
  \item Inventé en 1824
  \item Alphabet en relief
  \item 64 combinaisons pour tout faire
  \end{itemize}
\end{frame}

\begin{frame}  
  \frametitle{Le braille}
  \begin{block}{Exemple}
  \begin{center}
  \includegraphics[width=4cm]{./include/braille_alphabet.jpg}
  \end{center}
  Exemple du mot «alphabet» écrit en braille :
  \end{block}
\end{frame}


\begin{frame}  
  \frametitle{La langue des signes}
  \begin{itemize}
  \item IBM : Traduction texte vers langue des signes.
  \item Facilite la communication avec les malentendants.
  \item Grammaire assez floue.
  \item Pas exactement le même vocabulaire que le français.
  \item Encodage de mots et concepts.
  \end{itemize}
\end{frame}

\begin{frame}  
  \frametitle {Autres notations}
  \begin{block}{Les vues}
  \begin{itemize}
  \item Projections axonométriques (ISO 5456-3)
  \item Vue éclatée
  \end{itemize}
  \end{block}
  \begin{block}{Notation mathématique}
  \begin{center}
  \includegraphics[height=1.5cm]{./include/formule.jpg}
  \end{center}
  \end{block}
% TODO : une vue éclatée
% TODO : une formule
\end{frame}

\section{Conclusion}

\begin{frame}
  \frametitle{Conclusion}
  \begin{figure}
    \centering
    \begin{tikzpicture}[scale=0.5,node distance=0.5cm,font=\footnotesize]
      \node at (12,10) {};
      \draw[->] (0,0) -- (11cm,0);
      \draw[->] (0,0) -- (0,11cm);
      \node[anchor=north] at (10cm,0) {Expressivité};
      \node[anchor=south,xshift=0.5cm] at (0,11cm) {Formalisme};
      
      \only<2->{
        \node[fill=red,  fill opacity=0.5,circle, minimum width=0.5cm,minimum height=0.5cm,inner sep=1mm,rotate=0] (mb) at (1cm,10cm) {};
        \node[right=of mb,text width=2cm] (mbtext) {Morse\\Braille};
        \draw[draw=red,draw opacity=0.5,thick] (mbtext.170) -- (mb);
      }
      
      \only<3->{
        \node[fill=orange, fill opacity=0.5,ellipse,minimum width=1cm,minimum height=0.5cm,inner sep=1mm,rotate=-90] (sph) at (1cm,9.4cm) {};
        \node[anchor=west,at=(sph.30),xshift=1.5cm,yshift=-0.8cm,text width=2cm] (s) {Syllabaires\\Phonogrammes};
        \draw[draw=orange,draw opacity=0.5,thick] (s.170) -- (sph);
      }
      
      \only<4->{
        \node[fill=yellow!80!black, fill opacity=0.5,ellipse,minimum width=3cm,minimum height=0.7cm,inner sep=1mm,rotate=-65] (ipi) at (2cm,8cm) {};
        \node[anchor=north,at=(ipi.south),yshift=-1.5cm,xshift=5mm,text width=2cm,text centered] (i) {Idéogrammes\\Pictogramme\\LSF};
        \draw[draw=yellow!80!black,draw opacity=0.5,thick] (i) -- (ipi.-15);
      }
      
      \only<5->{
        \node[fill=green,fill opacity=0.5,ellipse,minimum width=2.8cm,minimum height=0.8cm,inner sep=1mm,rotate=-45] at (8.5cm,2.5cm) {Espéranto};
        \node[fill=blue, fill opacity=0.5,ellipse,minimum width=2.3cm,minimum height=0.7cm,inner sep=1mm,rotate=-20] at (8.5cm,1.5cm) {Français};
      }
      
      \only<6->{
        \node[fill=brown,  fill opacity=0.5,circle, minimum width=0.5cm,minimum height=0.5cm,inner sep=1mm,rotate=0] (no) at (6cm,10cm) {};
        \node[right=of no,text width=2cm] (notext) {Notations};
        \draw[draw=brown,draw opacity=0.5,thick] (notext) -- (no);

        \node[fill=brown,  fill opacity=0.5,circle, minimum width=0.5cm,minimum height=0.5cm,inner sep=1mm,rotate=0] (bcl) at (6cm,9cm) {};
        \node[right=of bcl,text width=2cm] (bcltext) {BCL};
        \draw[draw=brown,draw opacity=0.5,thick] (bcltext) -- (bcl);
      }
      
      \only<7->{
        \draw[draw=black,line width=1mm,draw opacity=0.5] (0.2,10.8) -- (10.8,0.2);
      }
    \end{tikzpicture}
    \caption{Relation entre expressivité et formalisme}
  \end{figure}
\end{frame}

\begin{frame}
  \frametitle{Sources}
  \begin{itemize}
  \item A formal analysis of a business contract language, Guido Governatori and Zoran Milosevic.
  \item {\small\url{http://ibm.com/press/us/en/pressrelease/22316.wss}}
  \item {\small\url{http://www.atelier-calligraphie.com/frames/savlogo.htm}}
  \end{itemize}
\end{frame}

\end{document}
