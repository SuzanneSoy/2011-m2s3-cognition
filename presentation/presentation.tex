\documentclass{beamer}
\usepackage[utf8]{inputenc}
\usepackage[frenchb]{babel}
\usepackage{tikz}
\usepackage[scriptsize]{caption}
\renewcommand*{\figurename}{}
\usetikzlibrary{shapes,positioning,snakes,calc,chains}
\usetheme{Darmstadt}
\usepackage{graphicx}

% \setbeamercolor{alerted text}{fg=blue}

\title{FMIN327 Cognition individuelle et collective\\ Protocoles artificiels entre agents naturels}
\author{DUPÉRON Georges \and\\ BONAVERO Yoann}
\institute{Université Montpellier II, Département informatique  \\ Master 2 IFPRU \\ Sous la direction de Monsieur Jacques Ferber}
\date{Jeudi, 3 novembre 2011}

\defbeamertemplate*{footline}{shadow theme}
{%
  \leavevmode%
  \hbox{\begin{beamercolorbox}[wd=.5\paperwidth,ht=2.5ex,dp=1.125ex,leftskip=.3cm plus1fil,rightskip=.3cm]{author in head/foot}%
    \usebeamerfont{author in head/foot}\insertframenumber\,/\,\inserttotalframenumber%\hfill\url{http://www.pticlic.fr/}
  \end{beamercolorbox}%
  \begin{beamercolorbox}[wd=.5\paperwidth,ht=2.5ex,dp=1.125ex,leftskip=.3cm,rightskip=.3cm plus1fil]{title in head/foot}%
    \usebeamerfont{title in head/foot}\insertshorttitle%
  \end{beamercolorbox}}%
  \vskip0pt%
}

\AtBeginSection[] { 
  \begin{frame}[plain] 
    \frametitle{Plan} 
    \tableofcontents[currentsection] 
  \end{frame} 
  \addtocounter{framenumber}{-1} 
}

\begin{document}
\renewcommand*{\figurename}{}

\begin{frame}
  \titlepage
\end{frame}

\section{Introduction}

\begin{frame}
  \begin{block}{}
    Pour se comprendre, il faut des concepts et un mécanisme d'échange communs.
  \end{block}
  \begin{figure}
    \centering
    \begin{tikzpicture}[scale=0.8,every node/.style={transform shape}]
      \begin{scope}
        \node[draw,rectangle,minimum width=3cm,minimum height=2cm,anchor=south west] (ra) at (0,0) {};
        \path[fill=black!50!white](0.7cm,0.7cm) circle (0.3cm);
        \path[fill=green!80!black]  (1.5cm,1.3cm) circle (0.3cm);
        \path[fill=orange] (2.3cm,0.7cm) circle (0.3cm);
        \node[draw,fill=green!30!red,rectangle,minimum width=0.3cm,minimum height=0.3cm] (la1) at (ra.15) {};
        \node[draw,fill=green!50!blue,rectangle,minimum width=0.3cm,minimum height=0.3cm] (la2) at (ra.-15) {};
      \end{scope}
      \begin{scope}[xshift=6cm]
        \node[draw,rectangle,minimum width=3cm,minimum height=2cm,anchor=south west] (rb) at (0,0) {};
        \path[fill=red]   (1cm,1cm) circle (0.3cm);
        \path[fill=blue]  (2cm,1cm) circle (0.3cm);
        \node[draw,fill=white!20!blue!70!red,rectangle,minimum width=0.3cm,minimum height=0.3cm] (lb1) at (rb.165) {};
        \node[draw,fill=green!50!blue,rectangle,minimum width=0.3cm,minimum height=0.3cm] (lb2) at (rb.195) {};
      \end{scope}
      \begin{scope}[xshift=3cm,yshift=3cm]
        \node[draw,rectangle,minimum width=3cm,minimum height=2cm,anchor=south west] (rc) at (0,0) {};
        \path[fill=red]   (0.7cm,0.7cm) circle (0.3cm);
        \path[fill=green!80!black] (1.5cm,1.3cm) circle (0.3cm);
        \path[fill=blue]  (2.3cm,0.7cm) circle (0.3cm);
        \node[draw,fill=white!20!blue!70!red,rectangle,minimum width=0.3cm,minimum height=0.3cm] (lc1) at (rc.270) {};
      \end{scope}
      
      \only<2->{\draw[<->, thick] (lb1)--(lc1);}
      \only<3->{\draw[<->, thick, dashed, red] (la2)--(lb2);}
      \only<4->{\draw[<->, thick, dashed, red] (la1)--(lc1);}
    \end{tikzpicture}
    \caption{Communication entre trois agents}
  \end{figure}
\end{frame}

\begin{frame}
\begin{block}{Définitions}
\begin{itemize}
	\item Protocole artificiel\\
	Règles arbitraires qui codifient l'échange entre des agents.\\
	Nous étudierons seulement les protocoles artificiels formels.
	\item Agent naturel\\
	Entité capable d'interagir non crée par l'homme.
\end{itemize}
\end{block}
\end{frame}

\section{Espéranto}

\begin{frame}
\begin{center}
\huge Espéranto
\end{center}
\begin{itemize}
\item Origine et but.
\item Une langue dite \emph{construite}
\item Une langue vivante.
\end{itemize}
\end{frame}

\begin{frame}
\begin{center}
\huge Espéranto
\end{center}
\begin{itemize}
\item Une grammaire régulière.
\\ Exemple eo + fr.
\item Une langue agglutinante.
\\ Exemple eo + fr.
\end{itemize}
\end{frame}

\section[Contrats]{Législation et contrats formels}

\begin{frame}  
  \begin{itemize}
  \item Buisness Contract Language (BCL)% GovMil.pdf
  \item Formalisation des obligations, permissions, pénalités.% en cas de violation des obligations
  \item Vérification de la consistance.
  \end{itemize}
\end{frame}

\section[Alphabets]{Alphabets et notations}

\begin{frame}
  {\Huge Logogrammes}
  \begin{itemize}
  \item Alphabet roman
  \item Alphabet grec
  \item \dots
  \end{itemize}
  % TODO : images
  \begin{itemize}
  \item Idéogrammes chinois
  \item Hiéroglyphes
  \item \dots
  \end{itemize}
\end{frame}

\begin{frame}
  {\Huge Le morse}
  \begin{itemize}
  \item Invention attribué à Samuel Morse (1791 - 1872)
  \item Créé en 1835
  \item Seulement deux types d'impulsion sont nécessaires.
  \end{itemize}
  % Ajouter la photo de Samuel Morse.
  
  \begin{figure}
    \centering
    \begin{tikzpicture}
      \begin{scope}[
        start chain,
        node distance=1mm,
        every node/.style={on chain},
        dot/.style={fill=black,circle,minimum height=1mm,minimum width=1mm,inner sep=0pt},
        dash/.style={fill=black,rounded rectangle,minimum height=1mm,minimum width=4mm,inner sep=0pt}]
        \node[dot] {}; \node[dash] {}; \node[dot] {}; \node[dash] {}; \node[dot] {}; \node[dot] {}; \node[dot] {}; \node[dash] {};
        \node[dash] {}; \node[dot] {}; \node[dot] {}; \node[dot] {}; \node[dot] {}; \node[dot] {}; \node[dot] {}; \node[dash] {};
        \node[dash] {}; \node[dot] {}; \node[dot] {}; \node[dot] {}; \node[dot] {}; \node[dash] {};
      \end{scope}
    \end{tikzpicture}
    \caption{Mot «Alphabet» écrit en morse.}
  \end{figure}
\end{frame}


\begin{frame}  
  \Huge{Le braille}
  \begin{center}
  % TODO
  \includegraphics[width=4cm]{./include/cellule_braille.jpg}
  \end{center}
  \normalsize \begin{itemize}
  \item Louis Braille (1809 - 1852)
  \item Inventé en 1824
  \item Alphabet en relief
  \item 64 combinaisons pour tout faire
  \end{itemize}
\end{frame}

\begin{frame}  
  \Huge Le braille
  \\ des exemples.
\end{frame}


\begin{frame}  
  { \Huge La langue des signes. }
  \begin{itemize}
  \item IBM : programme pour traduire texte vers Langue des signes.
  \item Facilite la communication avec les malentendants.
  \item Grammaire assez floue.
  \item Pas exactement le même vocabulaire que le français.
  \item Encodage de mots et concepts.
  \end{itemize}
\end{frame}

\begin{frame}  
  \begin{itemize}
  \item Vues perspective (ISO...) % TODO
  \item Vue éclatée
  \item Notation mathématique
  \end{itemize}
% TODO : une vue éclatée
% TODO : une formule
\end{frame}

\section{Conclusion}

% Graphique avec expressivité vs formel
% Esperanto : patate 7-10, 4-1
% Français : patate 7-10, 2-1
% Braille, morse : point 1, 10
% Alphabet roman : patate 1, 10-8
% Alphabets chinois etc. : patate 1-3, 10-6
% LSF : patate 1-3, 10-6
% Diagrammes et vues normalisées : point 6, 10
% BCL (contrats) : point 6, 9

\begin{frame}
\end{frame}

\end{document}
